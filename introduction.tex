\chapter{Introduction}\label{chap:introduction}
Bitcoin is the most prominent example of a decentralized cryptocurrency~\cite{1}. Before the development of Bitcoin a decentralized cryptocurrency had been envisioned for many years. It is a system, where a ledger is kept consistent among multiple parties in a peer-to-peer network without a trusted third party. It enables the deployment of electronic cash without a central authority figure like a bank.
For this reason it is an enhancement to the currently established electronic banking system.

Key to the system is a consistent and correct decentralized ledger. Without a central authority, the ledger must be solved in a cooperative, distributed manner.
Since multiple independent parties have to find agreement, the ledger is a Byzantine Agreement problem~\cite{garay2015bitcoin}.
Bitcoin assumes an honest majority in a public system~\cite{tschorsch}. Therefore, the consistence and correctness of the ledger can be formulated as a voting problem. However, voting in a public system is challenging, because of forged identities through sybil attacks~\cite{sybil}. Sybil attacks enable an attacker to obtain a dishonest majority. Bitcoin binds voting right to computational power through cryptographic puzzles. A peer has to solve such a puzzle in order to participate in the system. Through this mechanism Bitcoin is able to protect against sybil attacks.

This process, also known as mining, consumes computational resources of the peer. Mining is incentivised to achieve participation in the system. If a miner mines a block, he receives a mining reward. Without a mining reward there be no economical reason to spend computational resources on Bitcoin. Every party competes for mining rewards and as a result the overall computational power is spread across the system, which leads to a decentralized system. 



Mining is a process, which builds inherently on the idea of incentives. It is logical, that miners will strive for the best strategy to maximize rewards. A mining protocol maximizing rewards is called incentive compatible. It can also be assumed that a miner will always execute the mining protocol, which maximizes rewards. The original bitcoin mining protocol is assumed to be incentive compatible~\cite{1}. However, \citeauthor{eyal} show the existence of deviant mining protocols with greater rewards~\cite{eyal}. Miners executing such protocols are called selfish miners. This imposes a threat, since it reduces the performance of the overall system. Additionally, selfish miners obtain a greater voting power than their computational ressources allow and as a result tilt the honest majority balance.

The central goal of this master thesis is to analyze the impact of selfish mining as an attack on blockchain systems. 
While it has been established that selfish mining imposes a threat on blockchain, it remains unassessed how big the impact is. 
Additionally, selfish mining is highly influenced by networking effects. 
Therefore, in order to assess the impact of selfish mining, analysis has to be performed in a model, which also captures the underlying network.





