\chapter{Introduction}\label{chap:introduction}
Bitcoin is the most prominent example of a decentralized cryptocurrency~\cite{1}. Before the development of Bitcoin a decentralized cryptocurrency had been envisioned for many years. It is a system, where a ledger is kept consistent among multiple parties in a peer-to-peer network without the need of trust. It enables the deployment of electronic cash without a central authority figure like a bank.
For this reason it is an enhancement to the currently established electronic banking system.

 A consistent distributed ledger is essentially a consensus problem, which has to be solved in a cooperative, distributed manner. It is therefore a Byzantine Agreement problem~\cite{garay2015bitcoin}. Bitcoin assumes an honest majority in a public system~\cite{tschorsch}. Thus, the consistence and correctness of the ledger reduces to a voting problem. However, voting in a public distributed system remains a hard problem, especially considering sybil attacks~\cite{sybil}. Bitcoin reduces the effectiveness of sybil attacks by binding voting right to computational power.
In order for a peer to participate in the system, he has to solve a cryptographic puzzle.
This process, also known as mining, consumes the computational ressources of the peer. Since there would be no reason to waste computational ressources without gain, mining is incentivized. A miner receives a so called mining reward for mining a block. This incentivized process helps spreading the overall computational power of the network among multiple different parties, since every party is competing for mining rewards.
Since mining is inherently constructed through incentives, miners will strive for the best strategy to maximize rewards. \citeauthor{eyal} show the existence of deviant mining protocols with greater rewards. Miners executing such protocols are called selfish miners. This imposes a threat, since it reduces the performance of the overall system. Additionally selfish miners obtain a greater voting power than their computational ressources allow and as a result tilt the honest majority balance.

The central goal of this master thesis is to analyze the impact of selfish mining as an attack on blockchain systems. 
While it has been established that selfish mining imposes a threat on blockchain, it remains unassessed how big the impact is. 
Additionally, selfish mining is highly influenced by networking effects. 
Therefore, in order to assess the impact of selfish mining, analysis has to be performed in a model, which also captures the underlying network.



