\chapter{Contribution}\label{chap:contribution}
The central goal of this master thesis is to analyze the impact of selfish mining as an attack on blockchain systems. 
While it has been established that selfish mining imposes a threat on blockchain, it remains unassessed how big the impact is. 
Selfish mining is highly influenced by networking effects. 
Therefore, in order to assess the impact of selfish mining, analysis has to be performed in a model, which also captures the underlying network.
Therefore, the model proposed by \citeauthor{gopalan} has been enhanced to model selfish mining behaviour in \ref{selfishmodel}.
The relationship between selfish mining and networking effects can be charactized by a number of key questions.
Those questions can be split up in two groups.
The first group considers how the network influences selfish mining.
Key aspects include:
\begin{enumerate}
\item \citet{xiao_modeling} show in their model that revenue gain and profitability threshhold correlates to betweenness centrality. Does this correlation also show \ref{selfishmodel}?
\item Does a networking advantage increase revenue gain and profitability threshhold?
\item Does a certain network topology influence selfish mining effectiveness?
\end{enumerate}
The second group considers how selfish mining influences the network.
Key aspects include:
\begin{enumerate}
\item Does the network have a different throughput, if one peer is executing the selfish mining protocol?
\item Does the network have a different block propagation time, if one peer is executing the selfish mining protocol?
\item Does the network show different congestion peeks, if one peer is executing the selfish mining protocol?
\item If the network shows congestion peeks, does it correlate to certain actions the selfish miner is performing?
\end{enumerate}
´






